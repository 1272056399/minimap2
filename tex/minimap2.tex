\documentclass{bioinfo}
\copyrightyear{2017}
\pubyear{2017}

\usepackage{graphicx}
\usepackage{hyperref}
\usepackage{url}
\usepackage{amsmath}
\usepackage[ruled,vlined]{algorithm2e}
\newcommand\mycommfont[1]{\footnotesize\rmfamily{\it #1}}
\SetCommentSty{mycommfont}
\SetKwComment{Comment}{$\triangleright$\ }{}

\usepackage{natbib}
\bibliographystyle{apalike}
\usepackage{hyperref}

\DeclareMathOperator*{\argmax}{argmax}

\begin{document}
\firstpage{1}

\title[Long DNA sequence alignment with minimap2]{Minimap2: fast pairwise alignment for long DNA sequences}
\author[Li]{Heng Li}
\address{Broad Institute, 415 Main Street, Cambridge, MA 02142, USA}

\maketitle

\begin{abstract}
\section{Summary:} Minimap2 is a general-purpose mapper to align long noisy DNA
sequences against a large reference database. It targets query sequences of
1kb--100Mb in length with per-base divergence typically below 25\%. Minimap2 is
$\sim$30 times faster than many mainstream long-read aligners and achieves
higher accuracy on simulated data. It also employs concave gap cost and rescues
inversions for improved alignment around potential structural variations.

\section{Availability and implementation:}
\href{https://github.com/lh3/minimap2}{https://github.com/lh3/minimap2}

\section{Contact:} hengli@broadinstitute.org
\end{abstract}

\section{Introduction}

Single Molecule Real-Time (SMRT) sequencing technology and Oxford Nanopore
technologies (ONT) produce reads over 10kbp in length at an error rate
$\sim$15\%. Several aligners have been developed for such
data~\citep{Chaisson:2012aa,Li:2013aa,Liu:2016ab,Sovic:2016aa,Liu:2017aa,Lin:2017aa,Sedlazeck169557}.
They are usually five times as slow as mainstream short-read
aligners~\citep{Langmead:2012fk,Li:2013aa}. We speculated there could be
substantial room for speedup on the thought that 10kb long sequences should be
easier to map than 100bp reads because we can more effectively skip repetitive
regions, which are often the bottleneck of short-read alignment. We confirmed
our speculation by achieving approximate mapping 50 times faster than
BWA-MEM~\citep{Li:2016aa}.  \citet{Suzuki:2016} extended our work with a fast
and novel algorithm on generating detailed alignment, which in turn inspired us
to develop minimap2 towards higher accuracy and more practical functionality.

\begin{methods}
\section{Methods}

Minimap2 is the successor of minimap~\citep{Li:2016aa}. It uses similar
indexing and seeding algorithms, and furthers it with more accurate chaining
and the ability to produce detailed alignment.

\subsection{Chaining}

%\subsubsection{Chaining}
An \emph{anchor} is a 3-tuple $(x,y,w)$, indicating interval $[x-w+1,x]$ on the
reference matching interval $[y-w+1,y]$ on the query. Given a list of anchors
sorted by the ending reference position $x$, let $f(i)$ be the maximal chaining
score up to the $i$-th anchor in the list. $f(i)$ can be calculated with
dynamic programming (DP):
\begin{equation}\label{eq:chain}
f(i)=\max\big\{\max_{i>j\ge 1} \{ f(j)+d(j,i)-\gamma(j,i) \},w_i\big\}
\end{equation}
where $d(j,i)=\min\big\{\min\{y_i-y_j,x_i-x_j\},w_i\big\}$ is the number of
matching bases between the two anchors. $\gamma(j,i)>0$ is the gap cost. It
equals $\infty$ if $y_j\ge y_i$ or $\max\{y_i-y_j,x_i-x_j\}>G$ (i.e. the
distance between two anchors is too large); otherwise
\[
\gamma(j,i)=\gamma'(\max\{y_i-y_j,x_i-x_j\}-\min\{y_i-y_j,x_i-x_j\})
\]
In implementation, a gap of length $l$ costs $\gamma'(l)=\alpha\cdot
l+\beta\log_2(l)$. For $m$ anchors, directly computing all $f(\cdot)$ with
Eq.~(\ref{eq:chain}) takes $O(m^2)$ time. Although theoretically faster
chaining algorithms exist~\citep{Abouelhoda:2005aa}, they
are inapplicable to generic gap cost, complex to implement and usually
associated with a large constant. We introduced a simple heuristic to
accelerate chaining.

We note that if anchor $i$ is chained to $j$, chaining $i$ to a predecessor
of $j$ is likely to yield a lower score. When evaluating Eq.~(\ref{eq:chain}),
we start from anchor $i-1$ and stop the evaluation if we cannot find a better
score after up to $h$ iterations. This approach reduces the average time to
$O(h\cdot m)$. In practice, we can almost always find the optimal chain with
$h=50$; even if the heuristic fails, the optimal chain is often close.

%\subsubsection{Backtracking}
For backtracking, let $P(i)$ be the index of the best predecessor of anchor
$i$. It equals 0 if $f(i)=w_i$ or $\argmax_j\{f(j)+\eta(j,i)-\gamma(j,i)\}$
otherwise. For each anchor $i$ in the descending order of $f(i)$, we apply
$P(\cdot)$ repeatedly to find its predecessor and mark each visited $i$ as
`used', until $P(i)=0$ or we reach an already `used' $i$. This way we find all
chains with no anchors used in more than one chains.

%\subsubsection{Identifying primary chains}
In the absence of copy number changes, each query segment should not be mapped
to two places in the reference. However, chains found at the previous step may
have significant or complete overlaps due to repeats in the reference.
Minimap2 used the following procedure to identify \emph{primary chains} that do
not greatly overlap on the query. Let $Q$ be an empty set initially. For each
chain from the best to the worst according to their chaining scores: if on the
query, the chain overlaps with a chain in $Q$ by 50\% or higher percentage of
the shorter chain, mark the chain as secondary to the chain in $Q$; otherwise,
add the chain to $Q$. In the end, $Q$ contains all the primary chains. We did
not choose a more sophisticated data structure (e.g. range tree or k-d tree)
because this step is not the performance bottleneck.

\subsection{Alignment}

Minimap2 performs global alignment between adjacent anchors in a chain. It
adopted difference-based formulation to derive
alignment~\citep{Wu:1996aa,Suzuki:2016}. When combined with SSE vectorization,
this formulation has two advantages. First, because each score in the DP matrix
is bounded by the gap cost and the maximum matching score, we can usually
achieve 16-way SSE vectorization regardless of the peak score of the
alignment. Second, filling the DP matrix along the diagonal, we can simplify
banded alignment, which is critical to performance. In practice, our
implementation is three times as fast as Parasail's 4-way
vectorization~\citep{Daily:2016aa} for global alignment.
Without banding, our implementation is slower than Edlib~\citep{Sosic:2017aa},
but with a 1000bp band, it is considerably faster. When performing global
alignment between anchors, we expect the alignment to stay close to the
diagonal of the DP matrix. Banding is often applicable.

Minimap2 uses a 2-piece affine gap cost
$\gamma(l)=\min\{q+l\cdot e,\tilde{q}+l\cdot\tilde{e}\}$.
On the condition that $q+e<\tilde{q}+\tilde{e}$ and $e>\tilde{e}$, this
cost function is concave. It applies cost $q+l\cdot e$ to gaps shorter than
$\lceil(\tilde{q}-q)/(e-\tilde{e})\rceil$ and applies
$\tilde{q}+l\cdot\tilde{e}$ to longer gaps. This scheme helps to recover
longer insertions and deletions~(INDEL; \citealp{Gotoh:1990aa}).

With global alignment, minimap2 may force to align unrelated sequences between
two adjacent anchors. To avoid such an artifact, we compute accumulative
alignment score along the alignment path and break the alignment where the
score drops too fast in the diagonal direction. More precisely, let $S(i,j)$ be
the alignment score along the alignment path ending at cell $(i,j)$ in the DP
matrix. We break the alignment if there exist $(i',j')$ and $(i,j)$, $i'<i$ and
$j'<j$, such that
\[
S(i',j')-S(i,j)>Z+e\cdot(\max\{i-i',j-j'\}-\min\{i-i',j-j'\})
\]
where $e$ is the gap extension cost and $Z$ is an arbitrary threshold.
This strategy is similar to X-drop employed in BLAST~\citep{Altschul:1997vn}.
However, unlike X-drop, it would not break the alignment in the presence of a
single long gap.

When minimap2 breaks a global alignment between two anchors, it performs local
alignment between the two subsequences involved in the global alignment, but
this time with the one subsequence reverse complemented. This additional
alignment step may identify short inversions that are missed during chaining.

\end{methods}

\section{Results}
\begin{figure}[!tb]
\centering
\includegraphics[width=.5\textwidth]{roc-color.pdf}
\caption{Evaluation on simulated SMRT reads aligned against human genome
GRCh38. (a) ROC-like curve. (b) Accumulative mapping error rate as a function
of mapping quality. 33,088 $\ge$1000bp reads were simulated using
pbsim~\citep{Ono:2013aa} with error profile sampled from file
`m131017\_060208\_42213\_*.1.*' downloaded at
\href{http://bit.ly/chm1p5c3}{http://bit.ly/chm1p5c3}. The N50 read length is
11,628. A read is considered correctly mapped if the true position overlaps
with the best mapping position by 10\% of the read length. All aligners were
run under the default setting for SMRT reads.}\label{fig:eval}
\end{figure}

As a sanity check, we evaluated minimap2 on simulated human reads along with
BLASR~\citep{Chaisson:2012aa}, 
BWA-MEM~\citep{Li:2013aa},
GraphMap~\citep{Sovic:2016aa},
minialign~\citep{Suzuki:2016} and
NGMLR~\citep{Sedlazeck169557}. We excluded rHAT~\citep{Liu:2016ab},
LAMSA~\citep{Liu:2017aa} and Kart~\citep{Lin:2017aa} because they either
crashed or produced malformatted output. In this evaluation, Minimap2 has a
higher power to distinguish unique and repetitive hits, and achieves overall
higher mapping accuracy (Fig.~\ref{fig:eval}a). It is still the most accurate
even if we skip DP-based alignment (data not shown), suggesting chaining alone
is sufficient to achieve high accuracy for approximate mapping. Minimap2 and
NGMLR provide better mapping quality estimate: they rarely give repetitive hits
high mapping quality (Fig.~\ref{fig:eval}b).  Apparently, other aligners may
occasionally miss close suboptimal hits and be overconfident in wrong mappings.
On run time, minialign is slightly faster than minimap2. They are over 30 times
faster than the rest.  Minimap2 consumed 6.1GB memory at the peak, more than
BWA-MEM but less than others.

On real human SMRT reads, the relative performance and sensitivity of
these aligners are broadly similar to those on simulated data. We are unable to
provide a good estimate of mapping error rate due to the lack of the truth.  On
ONT ultra-long human reads~\citep{Jain128835}, BWA-MEM failed. Minialign and
minimap2 are over 70 times faster than others. We have also examined tens of
$\ge$100bp INDELs in IGV~\citep{Robinson:2011aa} and can confirm the
observation by~\citet{Sedlazeck169557} that BWA-MEM often breaks them into
shorter gaps. Minimap2 does not have this issue.

\section{Discussions}

Minialign and minimap2 are fast because a) with chaining, they can quickly
filter out most false seed hits~\citep{Li:2016aa} and reduce unsuccessful but
costly DP-based alignments; b) they implemented so far the fastest DP-based
alignment algorithm for long sequences~\citep{Suzuki:2016}. It is possible to
further accelerate minimap2 with a few other tweaks such as adaptive
banding~\citep{Suzuki130633} or incremental banding.

In addition to reference-based read mapping, minimap2 inherits minimap's
ability to search against huge multi-species databases and to find read
overlaps. On a few test data sets, minimap2 appears to yield slightly better
miniasm assembly. Minimap2 can also align long assemblies or closely related
genomes, though more thorough evaluations are needed. Genome alignment is an
intricate topic.

\section*{Acknowledgements}
We owe a debt of gratitude to Hajime Suzuki for releasing his masterpiece and
insightful notes before formal publication. We thank M. Schatz, P. Rescheneder
and F.  Sedlazeck for pointing out the limitation of BWA-MEM. We are also
grateful to early minimap2 testers who have greatly helped to fix various
issues.

\bibliography{minimap2}

\pagebreak
\appendix

\begin{methods}
\section*{Appendix}
A 2-piece gap cost function is
\[
\gamma(l)=\min\{q+l\cdot e,\tilde{q}+l\cdot\tilde{e}\}
\]
Without losing generality, we assume $q+e\le\tilde{q}+\tilde{e}$. The equation
to compute the optimal alignment under such a gap cost is~\citep{Gotoh:1990aa}
\begin{equation}\label{eq:ae86}
\left\{\begin{array}{l}
H_{ij} = \max\{H_{i-1,j-1}+s(i,j),E_{ij},F_{ij},\tilde{E}_{ij},\tilde{F}_{ij}\}\\
E_{i+1,j}= \max\{H_{ij}-q,E_{ij}\}-e\\
F_{i,j+1}= \max\{H_{ij}-q,F_{ij}\}-e\\
\tilde{E}_{i+1,j}= \max\{H_{ij}-\tilde{q},\tilde{E}_{ij}\}-\tilde{e}\\
\tilde{F}_{i,j+1}= \max\{H_{ij}-\tilde{q},\tilde{F}_{ij}\}-\tilde{e}
\end{array}\right.
\end{equation}
where $s(i,j)$ is the score between the $i$-th reference base and $j$-th query
base. If we define
\[
\left\{\begin{array}{ll}
u_{ij}\triangleq H_{ij}-H_{i-1,j} & v_{ij}\triangleq H_{ij}-H_{i,j-1} \\
x_{ij}\triangleq E_{i+1,j}-H_{ij} & \tilde{x}_{ij}\triangleq \tilde{E}_{i+1,j}-\tilde{H}_{ij} \\
y_{ij}\triangleq F_{i,j+1}-H_{ij} & \tilde{y}_{ij}\triangleq \tilde{F}_{i,j+1}-\tilde{H}_{ij}
\end{array}\right.
\]
we can transform Eq.~(\ref{eq:ae86}) to
\begin{equation}\label{eq:suzuki}
\left\{\begin{array}{lll}
z_{ij}&=&\max\{s(i,j),x_{i-1,j}+v_{i-1,j},y_{i,j-1}+u_{i,j-1},\\
&&\tilde{x}_{i-1,j}+v_{i-1,j},\tilde{y}_{i,j-1}+u_{i,j-1}\}\\
u_{ij}&=&z_{ij}-v_{i-1,j}\\
v_{ij}&=&z_{ij}-u_{i,j-1}\\
x_{ij}&=&\max\{0,x_{i-1,j}+v_{i-1,j}-z_{ij}+q\}-q-e\\
y_{ij}&=&\max\{0,y_{i,j-1}+u_{i,j-1}-z_{ij}+q\}-q-e\\
\tilde{x}_{ij}&=&\max\{0,\tilde{x}_{i-1,j}+v_{i-1,j}-z_{ij}+\tilde{q}\}-\tilde{q}-\tilde{e}\\
\tilde{y}_{ij}&=&\max\{0,\tilde{y}_{i,j-1}+u_{i,j-1}-z_{ij}+\tilde{q}\}-\tilde{q}-\tilde{e}
\end{array}\right.
\end{equation}
where $z_{ij}$ is a temporary variable that does not need to be stored. We can
see that
\[
x_{ij}=E_{i+1,j}-H_{ij}=\max\{-q,E_{ij}-H_{ij}\}-e
\]
With $E_{ij}\le H_{ij}$, we have
\[
-q-e\le x_{ij}\le\max\{-q,0\}-e=-e
\]
and similar inequations for $y_{ij}$, $\tilde{x}_{ij}$ and $\tilde{y}_{ij}$.
In addition,
\[
u_{ij}=z_{ij}-v_{i-1,j}\ge\max\{x_{i-1,j},\tilde{x}_{i-1,j}\}\ge-q-e
\]
As the maximum value of $z_{ij}=H_{ij}-H_{i-1,j-1}$ is $M$, the maximal
matching score, we can derive
\[
u_{ij}\le M-v_{i-1,j}\le M+q+e
\]
In conclusion, all values in Eq.~(\ref{eq:suzuki}) are bounded: $x$ and $y$ by
$[-q-e,-e]$ and $\tilde{x}$, $\tilde{y}$ by
$[-\tilde{q}-\tilde{e},-\tilde{e}]$, and $u$ and $v$ by $[-q-e,M+q+e]$. When
matching score and gap cost are small, each of them can be stored as a 8-bit
integer. This enables 16-way SSE vectorization regardless of the peak score of
the alignment.

For a more efficient SSE implementation, we transform the row-column coordinate
to diagonal-anti-diagonal coordinate by letting $r\gets i+j$ and $t\gets i$.
Eq.~(\ref{eq:suzuki}) becomes:
\begin{equation*}
\left\{\begin{array}{lll}
z_{rt}&=&\max\{s(t,r-t),x_{r-1,t-1}+v_{r-1,t-1},y_{r-1,t}+u_{r-1,t},\\
&&\tilde{x}_{r-1,t-1}+v_{r-1,t-1},\tilde{y}_{r-1,t}+u_{r-1,t}\}\\
u_{rt}&=&z_{rt}-v_{r-1,t-1}\\
v_{rt}&=&z_{rt}-u_{r-1,t}\\
x_{rt}&=&\max\{0,x_{r-1,t-1}+v_{r-1,t-1}-z_{rt}+q\}-q-e\\
y_{rt}&=&\max\{0,y_{r-1,t}+u_{r-1,t}-z_{rt}+q\}-q-e\\
\tilde{x}_{rt}&=&\max\{0,\tilde{x}_{r-1,t-1}+v_{r-1,t-1}-z_{rt}+\tilde{q}\}-\tilde{q}-\tilde{e}\\
\tilde{y}_{rt}&=&\max\{0,\tilde{y}_{r-1,t}+u_{r-1,t}-z_{rt}+\tilde{q}\}-\tilde{q}-\tilde{e}
\end{array}\right.
\end{equation*}
In this formulation, cells with the same row index $r$ are independent of each
other. This allows us to fully vectorize the computation of all cells on the
same anti-diagonal in one inner loop.
On the condition that $q+e<\tilde{q}+\tilde{e}$ and $e>\tilde{e}$, the boundary
condition of this equation in the diagonal-anti-diagonal coordinate is
\[
\left\{\begin{array}{l}
x_{r-1,-1}=y_{r-1,r}=-q-e\\
\tilde{x}_{r-1,-1}=\tilde{y}_{r-1,r}=-\tilde{q}-\tilde{e}\\
u_{r-1,r}=v_{r-1,-1}=\eta(r)\\
\end{array}\right.
\]
where
\[
\eta(r)=\left\{\begin{array}{ll}
-q-e & (r=0) \\
-e & (r<\lceil\frac{\tilde{q}-q}{e-\tilde{e}}-1\rceil) \\
r\cdot(e-\tilde{e})-(\tilde{q}-q)-\tilde{e} & (r=\lceil\frac{\tilde{q}-q}{e-\tilde{e}}-1\rceil) \\
-\tilde{e} & (r>\lceil\frac{\tilde{q}-q}{e-\tilde{e}}-1\rceil)
\end{array}\right.
\]

\citet{Suzuki:2016} first derived a similar set of equations under affine gap
cost but with different notations.
\end{methods}
\end{document}
